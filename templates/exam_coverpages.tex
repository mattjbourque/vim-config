
\begin{document}
\ifprintanswers
  \centerline{\large <+class number+> - <+semester+> - <+exam number+> Solutions}
\else
  \thispagestyle{empty}
  \begin{coverpages}

    \makebox[.9\textwidth]{Print name:\enspace\hrulefill}

    \bigskip

    \begin{center}
      {\Large <+class number+> <+semester+>}\\
      <+exam number+>  - <+date+>\\
      Instructor: Matt Bourque
    \end{center}

    \bigskip

    \noindent Show work where appropriate for all questions.
    No partial credit will be awarded without clear work, and where some computation is necessary for an answer no credit at all will be awarded for solutions with no work shown.
    % FIXME: the next line is meant for stats
    If you use a built-in calculator function, write down the command you use with its inputs, like \texttt{normalcdf(3.4, 1E99, 34, 6)}.

    \bigskip

    \begin{center} 
      \gradetable[v]

      \medskip

      The exam grade will be computed out of 150 points.
    \end{center}

    \vspace{\stretch{1}}

    \begin{center}
      \textbf{Do not turn this page over until the start of the exam is called.}
    \end{center}

  \end{coverpages}
\fi

